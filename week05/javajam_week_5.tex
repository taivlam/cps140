\documentclass[12pt,letterpaper]{article}
\usepackage[utf8]{inputenc}
\usepackage{amsmath}
\usepackage{amsfonts}
\usepackage{amssymb}

\usepackage[margin=1in]{geometry}

\title{JavaJam --- Week 5}
\date{October 9, 2020}
\author{Tai Lam \\ CPS 140}

\usepackage{setspace}
\doublespacing

\begin{document}
    
    \maketitle
    
    \section{Hands-On Practice Case}
    \subsection{Question 1}
    A hierarchical site organization is used for the JavaJam website, based on the site map in Figure 2.31.
    
    Yes, this is the most appropriate organization for the site because the JavaJam site is a commercial site, which needs to be organized.  (A \emph{linear} organization doesn't make sense, because the JavaJam site is not a tutorial, and the \emph{random} organization isn't appropriate either because JavaJam needs to be concise and straightforward the business's services.)
    
    \subsection{Question 2}
    \paragraph{Design Practices Well Implemented} Using the Best Practices Checklist in Table 5.1, there are three good design practices that have been implemented well.
    
    First, a general audience is being targeted.  This is indicated by the generally light or neutral background color and the black/neutral dark font colors used, which provide good contrast visually.
    
    Secondly, there is a consistent JavaJam logo and navigation header throughout the site (for the Home, Menu, and Music pages -- the Jobs page has not been created yet, so that is not being considered for this assignment).
    
    Third, through my own testing, the JavaJam site seems to have high browser compatibility.  Even though I currently only have Linux computers and Android devices, I was able to verify that the JavaJam site worked well on Firefox, Chromium (which is the source code base for Chrome),  Brave Browser (which is Chromium-based), and my Android smartphone running Bromite (a privacy-enhanced version of Chromium for Android).  Though I was not able to test JavaJam on Microsoft Edge, Safari, or Internet Explorer (which is largely deprecated with respect to official Microsoft support), I infer that there should be nothing about the JavaJam website that should prevent it from working on those browsers.
    
    \paragraph{Design Practices That Could Be Improved} There are three other design practices that could be improved.
    
    First, the footer could be slightly improved -- it does not have date or time indicated when the page was last updated.  (I am keeping in mind that nothing from Week 2-4 explicitly asked for the last updated date and time to be in the footer, but that's a good design choice to put into practice in general.)
    
    Second, (even though this topic will be covered in Chapter 7), the JavaJam site could be more mobile-friendly by using CSS and media queries to configure a responsive page layout for mobile devices.
    
    Lastly, the timeliness of the footer could be kept up to date - the last year of copyright is in 2018, yet it is 2020.  (I am aware that I'm supposed to just follow the book's directions for the JavaJam assignments, but that is something I noticed, according to the book's checklist, and there isn't all that much ``poor design" with the site, given what was covered during Weeks 2-5 so far.)
    
    \paragraph{Other Improvements} Some other ways I personally would like to improve the site is to offer a light and dark theme icon to toggle between the light and dark themes.
    
\end{document}

